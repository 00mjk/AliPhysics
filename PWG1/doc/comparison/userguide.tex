\documentclass{elsart}
\usepackage{epsfig,amsmath}

\begin{document}

\subsection{Reconstruction Performance}

A comparison of Monte-Carlo (MC) and reconstruction information is necessary to 
check the performance of the reconstruction algorithms. The PWG1 library has been started to make 
MC vs reconstruction comparison, to tune selection criteria (cuts) and to find the best representation 
of the data (observables, correlations between observables, parametrisations, etc.). The main goal is to 
get the best track reconstruction performance in the central barrel.   

\subsection{PWG1 library}

Current implementation of PWG1 allows us to run comparison for TPC and TPC-tracks.
The comparison of MC and reconstructed tracks is done in 4 steps: 

\begin{itemize}
\item[1.] Collection of MC information
\item[2.] Correlation of MC and reconstruction information
\item[3.] Filling of comparison objects
\item[4.] Analysis of comparison objects
\end{itemize}
 
In order to perform these steps the following 
components (classes) have been implemented in PWG1 library:

\begin{itemize}
\item The $AliGenInfoMaker$ collects MC information stored in 
 several simulation trees. It creates $AliMCInfo$, $AliMCKinkInfo$ and $AliMCV0Info$ objects for each MC track.  
\item The $AliRecInfoMaker$ correlates MC and reconstruction information (ESDs) and creates for each MC object (e.g. $AliMCInfo$) corresponding reconstruction object ($AliESDRecInfo$). 
\item In the third step, the comparison objects ($AliComparisonObject$)  are filled in $AliComparisonTask$ task. In this step the cuts ($AliMCInfoCuts$, $AliRecInfoCuts$) are applied. The comparison objects keep control histograms and cuts. 
\item Finally, the analysis of the comparison objects is done by using their own $Analyse()$ 
      functions. The result analyzed histograms can be saved for further studies.
\end{itemize}
 
The advantage of such analysis scheme is that the most time consuming
tasks (steps 1 and 2) can be run only once. It can be done during
ESD production. Then the comparison task can be run many times (e.g. on Proof) 
to tune selection criteria and to find the best representation of data. 

The basic components of the PWG1 library are: makers ($AliGenInfoMaker$, $AliRecInfoMaker$), 
tasks ($AliAnalysisTask$), cuts ($AliAnalysisCuts$) and comparison objects ($AliComparisonObject$). 

The makers derive from TObject ROOT class and have functionality to collect and correlate MC and reconstruction 
information for ITS, TPC, TRD and TOF detectors. They can be run locally/batch. The comparison tasks derive from $AliAnalysisTask$ and must implement its virtual methods. They can be run locally/batch, on Proof and on Grid. The cut objects derive from $AliAnalysisCut$ and must implement its virtual methods. The comparison objects derive from $AliComparisonObject$ base class and must implement its virtual methods.

\subsection{User implementation}

One has to implement the following classes to run its own analysis using PWG1 library:

\begin{itemize}
\item[1.] Implement comparison object which contains control histograms and its own cut object (if needed). 
\item[2.] Implement (if needed) cut object which contains all cuts which will be applied while filling comparison object. It is recommended to use $IsSelected(TObject*)$ method of the cut object while applying the cuts.
\end{itemize}
 
\subsection{Quick Start}
 
\begin{itemize}
\item[1.] Prepare input by running $AliGenInfoMaker$ and then $AliRecInfoMaker$ makers ($PWG1/Macros/RunMakers.C$)
\item[2.] Create comparison objects 
\item[3.] Create cut objects and pass them to corresponding comparison objects
\item[4.] Add comparison objects to comparison task \newline ($PWG1/Macros/RunAliComparisonTask.C$)
\item[5.] Run comparison analysis ($PWG1/Macros/RunGSI.C$)

\end{itemize}
 
\end{document}
