\documentclass{elsart}
\usepackage{epsfig,amsmath}
\begin{document}




\subsection{Tracking performance}


The tracking performance depend on various variables:
The most important are following:
\begin{itemize}
\item The track topology - detector contributing to measurement.
      The different topologies has different error parameterization, and the scale of error
      can be  significantly different. E.g DCA resolution for tracks refitted with the 
      ITS and TPC are by 2 orders of magnitude better than for TPC tracks only.
      
\item Multiple scaterring and energy loss are scaling with particle momenta and mean material budget.
      Typical -  behaviour smooth $p_{t}$ and $\theta$ scaling.
\end{itemize} 

The multiple scattering in material:
\begin{equation}
\sigma^2_\alpha=\frac{k. xOverX0}{\beta^2.p^2}
\end{equation}

The energy loss correction error:
\begin{equation}
\sigma^2_E= k. (\frac{k_{bb}}{\beta^2} + k)^2
\end{equation}
The energy loss correction is particularly important for low momenta tracks.


The track paramterization - 5 parameters:
\begin{itemize}
\item P0 - local y
\item P1 - local z
\item P2 - sinus   of local inclination angle $\phi$ in y direction
\item P3 - tangent of inclination angle $\theta$ in z direction
\item P4 - curvature  - 1/$p_t$
\end{itemize}

The error scaling:
\begin{equation}
\begin{split}
C=1/p_t \\
\sigma^2_{yy} \approx   \sigma^2_{yy0}+ k_p.C^2 \\
\sigma^2_{zz} \approx   \sigma^2_{zz0}+ k_p.C^2 \\
\sigma^2_{\phi\phi}     \approx  \sigma^2_{\phi\phi0}+ k_a.C^2 \\
\sigma^2_{\theta\theta} \approx  \sigma^2_{\theta\theta0}+ k_a.C^2 \\
\sigma^2_{cc} \approx  \sigma^2_{cc0}+ k_c.C^4\\
\end{split}
\end{equation}



\subsection{Resolution histograming and parameterization}

The resolution of the track parameters depends strongly on the momenta, 
respectivaly curvature. 
There is roughly linear scaling roughly linear scaling in position and 
angle, and parabolic scaling  for momentum resolution. The pread of the resolutions
because of the wide range of particle momenta is of the order of magnitude.
($p_t$ $<0.1,\inf>$, C $<0,10>$).

In order to save the space for histograms, the resolution can be appropriatetly 
scaled. Following scaling function are used:

\begin{equation}
\begin{split}
C=1/p_t \\
sf_{pp} =   \sqrt{s_{pp}+C}   \\
sf_{aa} =   \sqrt{s_{aa}+C}   \\
sf_{cc} =   \sqrt{s_{cc}+C^2} \\ 
\end{split}
\end{equation}
Choosing appropriate scaling coeficient s, the scaled resolution as function of the
particle curvature is roughly constant.  





\begin{thebibliography}{9}   % Use for  1-9  references

\bibitem{alice-ref}
ALICE tech. proposal CERN/LHCC 95-71.

\bibitem{aliroot-ref}
ALICE PPR v.1 CERN/LHCC 2003-049.

\bibitem{tpc-ref}
ALICE TPC TDR CERN/LHCC 2000-001.

\bibitem{its-ref} 
ALICE ITS TDR CERN/LHCC 99-12.

\bibitem{trd-ref} 
ALICE TRD TDR CERN/LHCC 2001-021.

\bibitem{tof-ref}
ALICE TOF TDR CERN/LHCC 2002-016.


\end{thebibliography}


\end{document}
