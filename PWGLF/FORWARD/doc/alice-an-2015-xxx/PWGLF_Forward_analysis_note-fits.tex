%% === Section on energy loss fits ===================================
\section{$\Delta$ Fits}
\label{sec:fits}

\FIXME{Example plots}
The single particle distribution of energy loss $x$ is best fit with a
Landau folded with a Gaussian \cite{nim:b1:16,phyrev:a28:615}.
\begin{align} 
  \label{eq:f}
 f(x;\Delta_p,\xi,\sigma') = \frac{1}{\sigma' \sqrt{2 \pi}}
 \int_{-\infty}^{+\infty} dx' f'_{L}(x',\Delta_p,\xi)
 e^{-\frac{(x-x')^2}{2\sigma'^2}}\quad,
\end{align}
where $ f'_{L}$ is the Landau distribution, $\Delta_p$ the most
probable energy loss, $ \xi$ the width of the Landau, and $
\sigma'^2=\sigma^2-\sigma_n^2 $.  Here, $\sigma$ is the variance of
the Gaussian, and $\sigma_n$ is a parameter modelling noise in the
detector.

For $i$ particles this is modified to 
\begin{align}
  f_i(x;\Delta_{p},\xi,\sigma')=f(x;\Delta_{p,i},\xi_i,\sigma_i')\quad,
\end{align}
corresponding to $ i$ particles i.e., with the substitutions 
\begin{align*}
  \Delta_p  &\rightarrow \Delta_{p,i} = i\left(\Delta_p + \xi\log(i)\right)\\
  \xi       &\rightarrow \xi_i       = i \xi\\
  \sigma    &\rightarrow \sigma_i    = \sqrt{i}\sigma\\
  \sigma'^2 &\rightarrow \sigma_i'^2 = \sigma_n^2 + \sigma_i^2
\end{align*}

Because of the convolution with a Gaussian, the most-probable-value
$\Delta_p'$ of the resulting distribution is not really at the
Landau most-probable-value $\Delta_p$.  In fact we find that
$\Delta_p' > \Delta_p$.

Ideally, one would find an analytic expression for this shift by
solving
\begin{align*}
  0 &= \frac{\text{d} f_i(x;\Delta_p,\xi,\sigma)}{\text{d}x}\nonumber\\
    &= \frac{\text{d}}{\text{d}x}\frac{1}{\sigma' \sqrt{2 \pi}}
     \int_{-\infty}^{+\infty} dx' f'_{L}(x',\Delta_p,\xi)
     e^{-\frac{(x-x')^2}{2\sigma'^2}}
\end{align*}
for $x$ as a function of $\Delta_p,\xi,\sigma,i$. However,
do to the complex nature of the Landau distribution this is not
really feasible.

Instead, the shift was studied numerically. Landau-Gauss
distributions for $i=1,\ldots$ where generated with varying
$\xi$ and $\sigma$.  The distributions was then numerically
differentiated and the root $\Delta_p'$ of that derivative
found numerically.  The difference
$\delta\Delta_p=\Delta_p'-\Delta_p$ was then studied as a
function of the $\sigma,\xi$ parameters and an approximate
expression was found
\begin{align}
  \delta\Delta_p \approx \frac{c \sigma u}{(1+1/i)^{p u^{3/2}}}
\end{align}
where $ u=\sigma/\xi$.  The parameters $c$ and $p$ is
found to depend on $ u$ only weakly, and for practical
applications where $u\approx1$, we set $ c=p=1/2$. 

For the evaluating the full energy loss distribution from
$1+2+\ldots,n$ particles, we evaluate
\begin{align}
  f_N(x;\Delta_p,\xi,\sigma',\mathbf{a})=\sum_{i=1}^N a_i
  f_i(x;\Delta_p,\xi,\sigma',a)\quad,
\end{align}
where $ f(x;\Delta_p,\xi,\sigma')$ is the convolution of a Landau with
a Gaussian, and $\mathbf{a}$ is a vector of weights for each $
f_i$. Note that $a_1 = 1$.

We fit the function 
$$
F_j(x;C,\Delta_p,\xi,\sigma,\mathbf{a}) = C
f_j(x;\Delta_p,\xi,\sigma',\mathbf{a})
$$
to energy loss distribution for a given $\eta$ bin for a sub-detector
in steps of increasing $j$.  The fit procedure is stopped and
$N_{\text{max}}=j$ when for
$j+1$: 
\begin{itemize}
\item the reduced $\chi^2$ exceeds a certain threshold (usually 20), or
\item the relative error $\delta p/p$ of any parameter of the fit
  exceeds a certain threshold (usually 0.12), or 
\item when the weight $a_{j+1}$ is smaller than some number (typically
  $10^{-5}$). 
\end{itemize}
The parameters $\Delta_p,\xi,\sigma,$ and $\mathbf{a}$ are stored for
later use in the density calculations (see
\secref{sec:sub:density_calculator}) and to define various cuts on
the energy loss (see \secref{sec:cuts}). 

To estimate the number of charge particles $n_t$ corresponding to a given
energy loss $\Delta_t$ in strip $t$ we can evaluate 
\begin{align}
  n_t &= G_{N_{\text{max}}}(\Delta_t;\Delta_p,\xi,\sigma,\mathbf{a}) =
  \frac{\sum_i^{N_{max}} i\,a_i\,f_i(\Delta_t;\Delta_p,\xi,\sigma)}{
    \sum_i^{N_{max}}\,a_i\,f_i(\Delta_t;\Delta_p,\xi,\sigma)}
  \tag{\ref{eq:nt}}\quad. 
\end{align}
%% Local Variables:
%%   TeX-master: "PWGLF_Forward_analysis_note.tex"
%%   ispell-dictionary: "british"
%% End:
