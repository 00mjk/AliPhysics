%% === AOD analysis ==================================================
\section{Building the final $\ndndeta$}
\label{sec:ana_aod}

To build the final $\ndndeta$ for a given event selection $X$ we loop
over the events stored in the \AOD{} and extract the per--event,
vertex dependent, exclusive or inclusive number of charged particles
per $\etaphi$ bin.   

\subsection{Event selection}

Generally, the event selection consists of 
\begin{enumerate}
\item Selecting events with a primary interaction that has a $z$
  coordinate $\IPz$ within some specified range --- typically
  $|\IPz|\le\unit[10]{cm}$. 
\item Rejecting events that have been flagged as containing pile--up.
  Typically, the \SPD{} tracklet pile--up flag is used. 
\item Rejecting events that have been flagged as an \SPD{} outlier
  event. 
\item Selecting events that have a given trigger type.
  \begin{itemize}
  \item A trigger type $X$ can be simple triggers like \texttt{MBOR},
    \texttt{MBOR} and at least one tracklet in the \SPD{} within
    $|\eta|<1$ (a.k.a.~\INELGT{}), or \texttt{V0AND}. 
  \item Alternatively, the trigger $X$ can depend on some event
    characteristic e.g., centrality or reference multiplicity. In that
    case, we select on the quantity and perform the analysis in bins
    of that quantity.
  \end{itemize}
\end{enumerate}

All standard centrality estimators are supported.  Furthermore, the
reference multiplicity\footnote{The number of unique tracks and
  tracklets within $|\eta|<0.8$.} is supported, as well as the special
\ppCol{} centrality estimators developed for the RunII
High--Multiplicity Task--Force.

\paragraph{\ppCol{} event classes and trigger efficiencies} 

When determining $\dndeta$ for physical event classes, such as
the full in--elastic cross--section, or the non--single--diffractive
events, in \ppCol{} we choose the underlying hardware and replayed
software trigger that gives the highest efficiency for that event
class.  \tabref{tab:dndeta:ppclasses} summarises the interesting
physical event classes in \ppCol{}, together with the underlying
trigger and its corresponding efficiency for each of the studied
collision energies.  Note, that the event class \INELGT{} is a pure
experimental event class, and as such the trigger efficiency is by
definition unity. 

\begin{table}[hbtp]
  \centering
  \caption{The \ppCol{} event classes and their corresponding
    underlying trigger and the efficiency \cite{pwgud:2015}.}
  \begin{tabular}{|cc|cccc|}
    \hline 
    \headColor%
    \textbf{Event} 
    & \textbf{Underlying} 
    & \multicolumn{4}{c|}{\textbf{Trigger efficiency}}\\
    \headColor%
    \textbf{class} 
    & \textbf{trigger} 
    & \multicolumn{4}{c|}{$\mathbf{\sqrt{s}}$}\\
    \headColor%
    &
    & \GeV[900]{}
    & \TeV[2.76]{} 
    & \TeV[7]{} 
    & \TeV[8]{}\\
    \hline
    \INEL{} & \texttt{MBOR} 
    & $0.91^{+0.03}_{-0.01}$ 
    & $0.88^{+0.06}_{-0.035}$ 
    & $0.85^{+0.06}_{-0.03}$ 
    & $0.85^{+0.06}_{-0.03}$\\
    \altRowColor{}%
    \INELGT{} & \texttt{MBOR} \& $N_{\text{tracklet},|\eta|<1}\ge1$ 
    & \multicolumn{4}{c|}{1} \\
    \NSD{} & \texttt{V0AND} 
    & $0.94\pm0.02$ 
    & $0.93\pm0.03$ 
    & $0.93\pm0.02$ 
    & $0.93\pm0.02$ \\
    \hline
  \end{tabular}
  \label{tab:dndeta:ppclasses}
\end{table}


\subsection{Summing $\dndeta$ and normalisation} 

To build the final $\dndeta$ distribution it is enough to sum
\eqref{eq:superhist} (\eqref{eq:superhist:2} if the secondary map
correction was turned off) and \eqref{eq:overflow} over all accepted
events, $\NA$, and correct for the acceptance
$\PhiAcc{}\,$\footnote{Since $\phiAcc$ is 1 or slightly less, we have
  that $\PhiAcc$ is roughly equal to (but not bigger than) $\NA$.}
\begin{align}
  \dndetadphi &= \sum_i^{\NA}\dndetadphi[i,\subIPz]\\ 
  \PhiAcc &= \sum_i^{\NA}\phiAcc\\
\intertext{and integrating over $\varphi$ gives}
  \ndndeta &=
  \frac{\NA}{\N{X}{}} \frac1{\Delta\eta} \int_0^{2\pi} \text{d}\varphi
  \frac{\dndetadphi}{\PhiAcc}\quad,\label{eq:eventnormdndeta}
\end{align}
where $\NA/\N{X}{}$ is the trigger normalisation.  The trigger
normalisation is given by 
\begin{align}
  \N{X}{} &= \frac{1}{\epsilon_X}\left[\NA +
    \alpha(\NnotV -
    \beta)\right]  \label{eq:fulleventnorm}\\
  & = \frac{1}{\epsilon_X}\left[\NA + \frac{\NA}{\NV}(\NT-\NV{} -
    \beta)\right]\nonumber \\
  & =\frac{1}{\epsilon_X}\NA\left[1+\frac{1}{\epsilon_V}-1-
    \frac{\beta}{\NV}\right]\nonumber\\
  & = \frac{1}{\epsilon_X}\frac{1}{\epsilon_V}\NA
  \left(1-\frac{\beta}{\NT{}}\right)\nonumber
\end{align}
where
\begin{description}
\item[$\epsilon_X$]  is the trigger efficiency for type
  $X\in\{\INEL,\INELONE,\NSD\}$ for \ppCol{} data and $\text{CENT}$
  for \PbPbCol{} and \pPbCol{}/\PbpCol{} data
\item[$\epsilon_V=\frac{\NV{}}{\NT{}}$] is the primary vertex
  efficiency evaluated over the data.
\item[$\NA$] is the number of events with a trigger \emph{and} a valid
  vertex in the selected range
\item[$\NV{}$] is the number of events with a trigger \emph{and} a valid
  vertex. 
\item[$\NT$] is the number of events with a trigger.
\item[$\NnotV{}=\NT-\NV{}$] is the number of events with a trigger
  \emph{but no} valid vertex
\item[$\alpha=\frac{\NA}{\NV}$] is the fraction of accepted events of
  the total number of events with a trigger and valid vertex.  
\item[$\beta=\N{a}{}+\N{c}{}-\N{e}{}$] is the number of background
  events \emph{with} a valid off-line trigger. This formula is the
  simplest case of one bunch crossing per trigger/background
  class. For more complicated collision setups the fractions in the
  formula change.
\end{description}
The two terms under the parenthesis in \eqref{eq:fulleventnorm} refers
to the observed number of event $\NA$, and the events missed because
of no vertex reconstruction.  Note, for $\beta\ll\NT{}$
%% \eqref{eq:fulleventnorm} 
this reduces to the simpler expression
$$
\N{X}{} = \frac1{\epsilon_X}\frac1{\epsilon_V}\NA{}
$$
The trigger efficiency $\epsilon_X$ for a given trigger type $X$ is
evaluated from simulations as
\begin{align*}
  \epsilon_X = \frac{\N{X\wedge \text{T}}{}}{\N{X}{}}\quad,
\end{align*}
that is, the ratio of number of events of type $X$ with a
corresponding trigger $T$ to the number of events of type $X$.

If the trigger $X$ introduces a bias on the measured number of events,
then \eqref{eq:eventnormdndeta} need to be modified to 
\begin{align}
  \ndndeta &= 
  \frac{\NA}{\N{X}{}}\frac1{\Delta\eta} \int_0^{2\pi} \text{d}\varphi
  \frac{1}{B\etaphi}\frac{\dndetadphi}{\PhiAcc}
  \label{eq:eventnormdndeta2}\quad,
\end{align}
where $B\etaphi$ is the bias correction.  This is typically
calculated from simulations using the expression 
\begin{align*}
  B\etaphi = \frac{\frac{1}{\N{X\wedge \text{T}}{}}\sum_i^{\N{X\wedge
        \text{T}}{}}
    \mult[,\text{primary}]\etaphi}{\frac{1}{\N{X}{}}\sum_i^{\N{X}{}}
    \mult[,\text{primary}]\etaphi}
\end{align*}
Note that typically $B\etaphi$ is unity.

%% --- Sub section on the empirical correction -----------------------
\subsection{The Empirical Correction} 
\label{sec:sub:empirical} 

The precision of the simulation based correction for secondary
particle production $\SecMap{}$ depends strongly on the accuracy with
which the material is specified in the simulation.  If there is not
enough or too much material (compared to the actual experiment)
specified in the simulations, we will calculate a too low or to high
value for $\SecMap$.  If the correct amount of material \emph{is}
present, but placed incorrectly, we will have a wrong distribution of
$\SecMap$.  Since there are many parts in the various detectors, some
of which are hard to describe, it is very unlikely that we will ever
reach a high enough accuracy so that $\SecMap$ is not associated with
very large uncertainties.

\begin{figure}[th!bp]
  \centering
  \figinput{satellite}
  \caption{Schematic drawing (not to scale) of the cross-section of
    the \ITS{}, \FMD{}, and \VZERO{} and the midpoints of the
    locations of the nominal and `satellite' interaction points. The
    long--dashed line designates a region of dense material designed
    to absorb all particles except $\mu^{\pm}$. The short-dashed line
    indicates the region of the \ALICE{} \ITS{}, which has dense
    material for its services on the surfaces near \FMD{2} and
    \FMD{3}. The area between \FMD{2}, \FMD{1} and \VZERO{}-A contains
    only the beryllium beam pipe. The dark grey shaded areas denote
    the paths particles would follow from $z=\unit[0]{\text{cm}}$ and
    $z=\unit[225]{\text{cm}}$ to \FMD{2} and \VZERO{}-A such that it
    is evident which material they would traverse
    \cite{Abbas:2013bpa}.}
  \label{fig:satellite_geom}
\end{figure}

\paragraph{Satellite Collisions} 
However, in the \PbPbCol{} running of 2010\footnote{The
  \texttt{LHC10h} period.} we were fortunate that the machine provided
collisions of beams at so--called \emph{satellite} locations.

These collisions happened at
$$
z \approx -187.5, -150, \ldots, -37.5, 37.5, 75, \ldots, 300,
\unit[337.5]{\text{cm}} \quad, 
$$ 
and were a consequence of the so-called `debunching effect'
\cite{maxime}.  Particles produced in these off--nominal interaction
point collisions pass through very different regions of \ALICE{}, as
illustrated in \figref{fig:satellite_geom}.  

As illustrated in \figref{fig:satellite_geom}, particles produced in
collisions at say $z\approx\unit[225]{\text{cm}}$ pass through a lot
less material than particles produced at the nominal interaction
region ($z\approx\unit[0]{\text{cm}}$), and the material they do pass
through is a lot simpler and much better described in the simulations.
This was exploited to produce the \ndndeta{} distributions for
\PbPbCol{} at \usNN{PbPb}{2760} for centrality bins $0-5-10-20-30\%$
\cite{Abbas:2013bpa}.  Another benefit of this approach was the
overlap between the \FMD{} and \VZERO{} acceptance, and that the
\VZERO{} and \FMD{} results could be checked against the results from
the \SPD{} tracklets.

\begin{figure}[h!tbp]
  \centering
  \subfigure[]{
    \label{fig:empirical:cent}\figinput[.45\linewidth]{empirical_cent}}
  \subfigure[]{
    \label{fig:empirical:secmap}\figinput[.45\linewidth]{secmap_empcor}}
  \caption{\subref{fig:empirical:cent} The empirical correction
    $\EmpCor[c]{}$ for 4 centralities, and the mean of these.  The
    lower plot shows the ratio of $\EmpCor[c]{}$ to the mean.  From
    this we see we $\pm2\%$ systematic error on
    $\EmpCor{}$. \subref{fig:empirical:secmap} The average empirical
    correction compared to simulation based
    $\langle\SecMap\rangle=1/2\pi\int\text{d}\varphi\SecMap{}$ for
    given $\IPz{}$ bin. } 
\end{figure}


\paragraph{Nominal interactions} 
We can then turn the argument around to form a new correction for
secondaries.  We perform the inclusive (primaries \emph{and}
secondaries) analysis for the pseudo--rapidity density
(\ndndeta{}$|_{\text{nominal,inclusive}}$) for \PbPbCol{} at
\usNN{PbPb}{2760} at nominal interaction points
($z\in[\unit[-10]{\text{cm}},\unit[+10]{\text{cm}}]$) by turning off
the secondary correction $\SecMap$ (see
\secref{sec:sub:sub:secmap}).The \emph{empirical correction} is then
defined as

\begin{equation}
  \label{eq:empirical:cent}
  \EmpCor[c]{} = \frac{%
    \left.\ndndeta\right|_{\text{nominal,inclusive}}}{%
    \left.\ndndeta\right|_{\text{satellite}}}\quad,
\end{equation}
where the denominator is the result from the satellite \PbPbCol{} at
\usNN{PbPb}{2760} analysis \cite{Abbas:2013bpa}, and the numerator is
the result from collisions at the nominal interaction point
\emph{without} the secondary correction.   The sub-script $c$
indicates that we can form this ratio separately in each of the 4
centrality bins we have satellite results from.  The result can be
seen in \figref{fig:empirical:cent}. 

\paragraph{Stability}
We see that the correction is stable over all centralities and that we
can use the weighted average over all centralities to define 

\begin{equation}
  \label{eq:empirical}
  \EmpCor{} = \frac{\sum_c \Delta c\EmpCor[c]{}}{\sum_c \Delta c}\quad.
\end{equation}

Further motivation for the empirical correction as a correction
for the produced secondary corrections, is found in
\figref{fig:empirical:secmap}.  Here, the simulation based $\SecMap$
is compared to the data driven $\EmpCor$.  We see that $\EmpCor$
follows the shape of $\SecMap$ overall with differences at $\eta<0$
and around $\eta=2$ -- as we would expect from a correction that
depends on the material distribution. 

\begin{figure}[h!tbp]
  \centering
  \subfigure[]{
    \label{fig:empirical:compare_es}\figinput[.55\linewidth]{compare_es}}
  \subfigure[]{
    \label{fig:empirical:es_pp0900}\figinput[.35\linewidth]{es_pp0900}}
  \caption{\subref{fig:empirical:compare_es} Comparison of $\EmpCor$ and
    $\EmpCor[\subIPz,r]{}$.  The spread of $\EmpCor[\subIPz,r]{}$
    reflects the different material seen at different values of
    $\IPz$.   \subref{fig:empirical:es_pp0900} Comparison of applying
    either $\EmpCor$ or $\EmpCor[\subIPz,r]{}$ to \ppCol{} at
    \usNN{pp}{0900}. }
  \label{fig:empirical:methods}
\end{figure}

In principle we could form the ratio of \eqref{eq:empirical:cent} in
separate $\IPz$ bins and separately for each \FMD{} ring.  We would
then apply that correction $\EmpCor[\subIPz,r]{}$ instead of $\SecMap$
as outlined in \secref{sec:sub:sub:secmap}.  A comparison of the two
methods is shown in \figref{fig:empirical:methods}.  We see that the
difference in using either the single correction $\EmpCor$ or the
interaction point, ring dependent correction $\EmpCor[\subIPz,r]{}$ is
less than 2\%.

\begin{figure}[h!tbp]
  \centering
  \figinput[.9\linewidth]{empcor_stable}
  \caption{Stability of $\EmpCor$. The top panel shows the empirical
    correction using different runs for the inclusive reference
    (numerator of \eqref{eq:empirical:cent}). At the bottom, the ratio
    to the canonical reference run (138190) is shown, as well as the
    average distance to that canonical reference. Two groups are
    clearly seen: Before run 138190 and after run 138190.  From the
    Quality Assurance we know that the runs before 138190 are
    problematic, and are therefore not considered. }
  \label{fig:empirical:empcor_stable}
\end{figure}

The empirical correction shown in \figref{fig:empirical:methods} comes
from analysing a single run (138190) to obtain the numerator of
\eqref{eq:empirical:cent}.  To access the stability of the correction
with respect to the reference data, all of the \PbPbCol{} data from
2010 was analysed.  The result is shown in
\figref{fig:empirical:empcor_stable}.  Apart from the problematic runs
before run 138190, the resulting corrections all fall within
$\pm5\%$. 


\paragraph{Universality} The empirical correction $\EmpCor$ as a
correction for secondary particle production is by nature
\emph{universal}.  That is, it can be applied to \emph{any} collision
system and energy since it depends only on the material in \ALICE{}.
Howver, as shown in \figref{fig:secmap:all} there is a residual
uncertainty from the hadron chemistry which especially effects
\FMD{1}.  This forces us to assign a systematic error of the order of
a few percent more in the \FMD{1} region than what we would otherwise
to in the remaining regions. 
\FIXME{Marek, if your studies show we can ignore this, then this
  should be reformulated}


The empirical correction is seen to be stable and known to within
$\pm5\%$.  


\paragraph{The final \ndndeta{}}
We can then apply this correction to the inclusive pseudo--rapidity
density (\indndeta{}) for \emph{any} system and collision energy to
find the \emph{primary} charged particle pseudo--rapidity density

\begin{equation}
  \label{eq:empirical:applied}
  \ndndeta = \frac{1}{\EmpCor{}} \indndeta{}
\end{equation}

\subsection{Systematic Uncertainties}
\label{sec:sub:dndeta:syserr}

\tabref{tab:syserr} summarises the sources of the systematic
uncertainties and their sizes.


\begin{table}[h!tbp]
  %% Although this note really only deals with PbPb, I think we should
  %% show all we know here. 
  \centering
  \caption[Summary of systematic errors]{Summary of systematic
    errors. Errors on the 
    event selection in \ppCol{} are taken from \cite{pwgud:2015}.
    Errors on the centrality selection in \pPbCol{} and \PbpCol{} from
    \cite{Adam:2014qja}.  \PbPbCol{} systematics from
    \cite{PbPbCent:XXX}. \newline
    %% 
    \textsuperscript{*}These
    entries are not directly calculated into the total. \newline
    \textsuperscript{**}These are estimated from the difference to the
    empirical correction.}
  \footnotesize
  \begin{tabular}[t]{|cc|c|cccc|cc|c|}
    \hline
    \headColor%
    \multicolumn{2}{|l|}{\textbf{System}} 
    & Corre- 
    & \multicolumn{4}{c|}{\ppCol{}} 
    & \pPbCol{} 
    & Pb-p 
    & \PbPbCol{}\\
    \headColor%
    \multicolumn{2}{|l|}{\textbf{Source}} 
    & lated 
    & \GeV[900]{}
    & \TeV[2.76]{} 
    & \TeV[7]{}
    & \TeV[8]{}
    & \multicolumn{2}{c|}{\TeV[5.02]{}}
    & \TeV[2.76]{}\\
    \hline
    & \INEL   
    & y 
    & ${}_{-0.1}^{+0.3}\%$ 
    & ${}_{-0.35}^{+0.6}\%$ 
    & ${}_{-0.3}^{+0.6}\%$ 
    & ${}_{-0.3}^{+0.6}\%$ 
    & \multicolumn{2}{c|}{n/a}
    & n/a \\
    \altRowColor \cellcolor{white}
    & \INELGT 
    & y 
    & \multicolumn{4}{c|}{negl.} 
    & \multicolumn{2}{c|}{n/a} 
    & n/a \\
    & \NSD    & y & $\pm2\%$ 
    & $\pm3\%$  
    & $\pm2\%$ 
    & $\pm2\%$ 
    & 1--2\% & 1--2\% & n/a \\
    \altRowColor\cellcolor{white}
    \multirow{-4}{*}{\minitab[c]{\rowcolor{white}Event\\ selection}} 
    & Centrality 
    & y 
    & \multicolumn{4}{c|}{n/a} 
    & 2--4\% 
    & 3--6\% 
    & 1--2\% \\
    \hline 
    & Merging 
    & n 
    & \multicolumn{4}{c|}{1\%} 
    & \multicolumn{2}{c|}{1\%} 
    & 1\%\\ 
    \altRowColor\cellcolor{white}
    \multirow{-2}{*}{Analysis} 
    & Density 
    & n 
    & \multicolumn{4}{c|}{1\%} 
    & \multicolumn{2}{c|}{1\%} 
    & 1\%\\ 
    \hline
    \multicolumn{2}{|c|}{Secondary map\textsuperscript{*}\hspace*{4em}} 
    & n 
    & \multicolumn{7}{c|}{5-15\%\textsuperscript{**}}\\
    \hline
    \altRowColor{}\cellcolor{white} 
    & Satellite\textsuperscript{*} 
    & n 
    & \multicolumn{7}{c|}{3.5-7.5\%} \\
    & Reference\textsuperscript{*}  
    & n 
    & \multicolumn{4}{c|}{4\%} 
    & \multicolumn{2}{c|}{4\%} 
    & 5\%?\\ 
    \altRowColor{}\cellcolor{white} 
    \multirow{-3}{*}{Empirical} 
    & As--applied
    & n 
    & \multicolumn{4}{c|}{6.1\%} 
    & \multicolumn{2}{c|}{6.1\%} 
    & 6.1\%\\ 
    \hline
    \multicolumn{2}{|c|}{Hadron chemistry\hspace*{4em}}
    & n 
    & \multicolumn{4}{c|}{2\%(?)} 
    & \multicolumn{2}{c|}{2\%(?)} 
    & 2\%\\ 
    \hline
    \altRowColor\cellcolor{white}
    &   \INEL{}
    & 
    & \multicolumn{4}{c|}{6.3\% (6.6\%)}
    & \multicolumn{2}{c|}{n/a}
    & n/a 
    \\
    & \INELGT{}
    & 
    & \multicolumn{4}{c|}{6.3\% (6.6\%)}
    & \multicolumn{2}{c|}{n/a}
    & n/a \\
    \altRowColor\cellcolor{white}
    & \NSD{}
    & 
    & \minitab[c]{6.6\%\\(6.9\%)}
    & \minitab[c]{6.9\%\\(7.2\%)}
    & \multicolumn{2}{c|}{\minitab[c]{6.6\%\\(6.9\%)}}
    & \multicolumn{2}{c|}{\minitab[c]{6.6\%\\(6.9\%)}}
    & n/a \\
    \multirow{-4}{*}{Total} 
    & Centrality 
    & 
    & \multicolumn{4}{c|}{n/a} 
    & \minitab[c]{6.6--7.4\%\\(6.9--7.7\%)}
    & \minitab[c]{6.9--8.7\%\\(7.2--8.9\%)}
    & \minitab[c]{6.3--6.6\%\\(6.6--6.9\%)} \\
    \hline
  \end{tabular}
  \label{tab:syserr}
\end{table}

\begin{description}
\item[Event selection] \hbox{}\hfill
  \begin{description}
  \item[\INEL{} \& \NSD{}] The source of this systematic error is
    mainly the estimated error on the trigger selection, as well as
    the estimated error on the vertex reconstruction.  Other sources
    include pile-up.  Of these sources, the most dominant is the
    trigger efficiency and the rest are considered negligible
    \cite{pwgud:2015}.
  \item[\INELGT{}] Since this trigger is purely experimentally defined
    there is not or very small errors associated with the trigger
    efficiency, and the remaining contributions from vertex
    reconstruction and pile-up are negligible \cite{pwgud:2015}.
  \item[Centrality] The ranges given here are from most central (lower
    systematic error) to most peripheral (higher systematic error)
    \cite{Adam:2014qja,PbPbCent:XXX}.
  \end{description}
\item[Analysis] \hbox{}\hfill
  \begin{description}
  \item[Merging] This comes from the process of merging signals that
    are shared over 2 strips (see \secref{sec:sub:sharing_filter}).
    This has been evaluated by varying cuts and cut definitions (see
    \secref{sec:cuts}). The cuts were varied within the ranges that
    maintained the signal integrity.
  \item[Density] This comes from the estimation of the inclusive
    number of charge particles per $\etaphi$--bin.  The size of this
    error is estimated by evaluating the correlation between the two
    available methods for obtaining $\dndetadphi[incl,r,\subIPz,i]$
    (see \secref{sec:sub:density_calculator}), and by varying the
    threshold cut and cut definitions (see \secref{sec:cuts}).  The
    cuts were varied within the ranges that maintained the signal
    integrity. 
  \end{description}
\item[Secondary map] This row is mainly included for historical
  reasons and apply only in the case where the $\SecMap$ correction is
  applied to the data (see \secref{sec:sub:sub:secmap}). 
\item[Empirical] \hbox{}\hfill 
  \begin{description}
  \item[Satellite] These are the systematic errors from the published
    $\ndndeta$ based on an analysis of satellite collisions
    \cite{Abbas:2013bpa}.  The variation in the number stems from the
    uncertainty in the centrality determination, which presumably
    biases both the denominator and numerator of
    \eqref{eq:empirical:cent} in the same direction. The effect on the
    empirical correction is discussed in \secref{sec:dndeta:sys:emp}. 
  \item[Reference] As shown in \figref{fig:empirical:cent} and
    \ref{fig:empirical:methods} the variation of chosen reference run
    and the choice of method results in a 5\% systematic error. The
    effect on the empirical correction is discussed in
    \secref{sec:dndeta:sys:emp}.
  \item[As--applied] This is the systematic error as applied to the
    data in the this analysis. The error is discussed in
    \secref{sec:dndeta:sys:emp}.
  \end{description}
\item[Hadron Chemistry] An additional 2\% in the \FMD{1} region should
  be added in quadrature and stems from the additional sensitivity to
  the hadron chemistry discussed elsewhere in
  \secref{sec:sub:sub:secmap}.
\FIXME{Note that the systematic error on the empirical correction
  already contain 2-3\% error on this in the FMD region --- see \tabref{tab:satellite:syserrs}.} 
\end{description}

\subsubsection{Systematics error on the Empirical Correction} 
\label{sec:dndeta:sys:emp}

The \ndndeta{} in centrality bins from 0 to 30\% previously published
\cite{Abbas:2013bpa} is based on three measurements: \SPD{} tracklets,
\VZERO{} amplitude match to the \SPD{}, and the \FMD{} signals.  Each
of these measurements carried their own set of systematic errors, as
well as some common ones. \tabref{tab:satellite:syserrs} summarises
the various contributions. 

\begin{table}[htbp]
  \centering
  \caption{The systematic errors of the published \ndndeta{}.  Adapted
    from \cite{Abbas:2013bpa}.}
  \begin{tabular}[T]{|clc|}
    \hline 
    \headColor
    \textbf{Detector} 
    & \textbf{Source} 
    & \textbf{Error}\\ 
    \hline 
    Common 
    & Centrality 
    & 1--2\%\\
    \hline 
    % SPD 
    & Background subtraction   
    & 0.1--2\%\\
    \altRowColor{}\cellcolor{white} % SPD    
    & Particle composition
    & 1\%\\
    % SPD
    & Weak decays 
    & 1\%\\
    \altRowColor{}\cellcolor{white}\multirow{-4}*{\SPD}
    & Extrapolation to $p=0$ 
    & 2\%\\                              
    \hline 
    & Material budget 
    & 4\%\\
    \altRowColor{}\cellcolor{white}
    \multirow{-2}*{\minitab[c]{%
    \cellcolor{white}\FMD{}\\
    \cellcolor{white}\& \VZERO{}}} 
    & ZEM scaling 
    & 4\% \\    
    \hline
    & Particle composition, spectra, weak decays 
    & 2\%\\
    \altRowColor{}\cellcolor{white} 
    & Variation of cuts 
    & 3\%\\
    \multirow{-3}*{\FMD{}} 
    & Analysis method 
    & 2\%\\ 
    \hline
    \altRowColor{}\cellcolor{white} 
    & Variation between rings 
    & 3\%\\
    \multirow{-2}*{\VZERO{}}
    & Calibration by \SPD{} 
    & 3--4\%\\
    \hline 
  \end{tabular}
  \label{tab:satellite:syserrs}
\end{table}

The 3 measurements was then added up using the non--common systematic
errors as weights, and finally the common systematics errors are added
in quadrature.  The result is shown in the top of
\figref{fig:dndeta:sat:results}. Subtracting the common error from the
centrality estimate from the systematic error leaves us with an error
of 7.2\% in the high limit, and 3.6\% in the low limit.
 
\begin{figure}[h!tbp]
  \centering
  \figinput[.8\linewidth]{satellites_dndeta}
  \caption{Top shows the \ndndeta{} from satellite collisions after
    summing the 3 measurements, and adding the common systematics in
    quadrature \cite{Abbas:2013bpa}.  The bottom panel shows the
    result relative systematic error for each centrality bin. For the
    points relevant for the forward analysis we see a variation from
    3.75\% --- for the most central --- up to 7.5\% --- for the most
    peripheral.  The dashed line indicates the centrality weighted
    mean. }
  \label{fig:dndeta:sat:results}
\end{figure}

As outlined in \secref{sec:sub:empirical}, the empirical correction is
formed by averaging over the available centralities 
\begin{equation}
  \EmpCor{} = \frac{\sum_c \Delta c\EmpCor[c]{}}{\sum_c \Delta c}
  \tag{\ref{eq:empirical}}
\end{equation}
where $\EmpCor[c]{}$ is given by \eqref{eq:empirical:cent}.  That
means, we should consider the centrality weighted systematic error (as
shown in \figref{fig:dndeta:sat:results}), and we find a low value of
$\sqrt{4.2^2-1.5^2}=3.9\%$ and a high value of
$\sqrt{4.9^2-1.5^2}=4.7\%$, with an average of
$\sqrt{4.5^2-1.5^2}=4.2$ in the $\eta$ regions relevant for this
analysis. 

The error on the numerator of \eqref{eq:empirical:cent} is given by
the run--to--run variation illustrated in
\figref{fig:empirical:empcor_stable}, and amounts to 4\%.  Adding this
up in quadrature gives the final systematic on the empirical
correction to be 
\begin{align}
  \label{eq:empcorr:syserr}
  \min{\delta\EmpCor{}} &= \sqrt{3.9^2+4^2} = 5.6\%\nonumber\\
  \max{\delta\EmpCor{}} &= \sqrt{4.7^2+4^2} = 6.1\%\nonumber\\
  \langle\delta\EmpCor{}\rangle &= 5.8\%\quad.\\
\end{align}
Considering the overall variation across $\eta$ is relatively small,
we err on the side caution and choose $\delta\EmpCor{}=6.1\%$ for the
systematic error of empirical correction. 
%% Local Variables:
%%   TeX-master: "PWGLF_Forward_analysis_note.tex"
%%   ispell-dictionary: "british"
%% End:
